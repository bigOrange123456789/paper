\documentclass[11pt]{ctexart}  
\usepackage[top=1cm, left=1cm, right=1cm]{geometry}  
\usepackage{algorithm}  
\usepackage{algorithmicx}  
\usepackage{algpseudocode}  
\usepackage{amsmath}  

\floatname{algorithm}{算法}  
\renewcommand{\algorithmicrequire}{\textbf{输入数据:}}   
\renewcommand{\algorithmicensure}{\textbf{输出数据:}}  

\begin{document}  
	\begin{algorithm}  
		\caption{vertexfragment}  
		\begin{algorithmic} %每行显示行号  
			
			\Require 	
			\State
			vertexInf=\{ $position$,$UV$,$skinIndex$   \}//当前顶点的信息 \\
			instanceObjInf=\{  //实例化对象的信息\\
				\qquad $matrix$,\\
				\qquad $animationStyle$,//动画的类型和速度\\
				\qquad $textureStyle$,//贴图的类型和色调\\
			\}
			\State
			uniformInf=\{  //所有实例化对象共用的数据\\
			\qquad $skelontonDate$,//骨骼数据\\
			\qquad $time$,//时间用于计算当前的帧序号\\
			\}
			
			 \State   
			
			  
			  \State $judgeArea();$ //判断当前顶点处于身体的哪个部位
			  \State  $matrix1=computeAnimationMatrix();$ 
			  
			  %matrix2
			  \If {$bone(vertexInf.skinIndex) have \; animation$}//对应的骨骼有动画
			  \State $frameIndex=time*speed \ mod (frameIndexMax+1);$
			  \State $address0=addressGet1(skinIndex,type,frameIndex);$
			  \Else  //对应的骨骼没有动画
			  \State $address0=addressGet2(skinIndex,type);$
			  \EndIf   
			  \State $matrix2=getMatrix(skelontonDate,address0);$
			  
			  
			  
			  \State  $glPosition = modelViewProjectionMatrix*matrix2*matrix1*position;$
			  
			

			  
			
			\Ensure
			\State
			glPosition //当前顶点对应屏幕上的位置 \\
			sendFragmentShader=\{  //传递给FragmentShader的信息\\
			\qquad $areaType$,//当前点所在的区域\\
			\qquad $UV$,
			\qquad $textureType$,//贴图类型\\
			\qquad $color$,//由于色调调整\\
			\}
			
		\end{algorithmic}  
	\end{algorithm}  
\end{document}   