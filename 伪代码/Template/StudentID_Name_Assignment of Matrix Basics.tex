\documentclass{article}

\usepackage{fancyhdr}
\usepackage{extramarks}
\usepackage{amsmath}
\usepackage{amssymb}
\usepackage{amsthm}
\usepackage{amsfonts}
\usepackage{tikz}
\usepackage[plain]{algorithm}
\usepackage{algpseudocode}
\usepackage{enumerate}
\usetikzlibrary{automata,positioning}

%
% Basic Document Settings
%

\topmargin=-0.45in
\evensidemargin=0in
\oddsidemargin=0in
\textwidth=6.5in
\textheight=9.0in
\headsep=0.25in

\linespread{1.1}

\pagestyle{fancy}
\lhead{\hmwkAuthorName}
\chead{\hmwkClass}
\rhead{\hmwkTitle}
\lfoot{\Address}
\rfoot{Page \thepage}
\cfoot{}

\renewcommand\headrulewidth{0.4pt}
\renewcommand\footrulewidth{0.4pt}

\setlength\parindent{0pt}

%
% Create Problem Sections
%

\newcommand{\enterProblemHeader}[1]{
    \nobreak\extramarks{}{Problem \arabic{#1} continued on next page\ldots}\nobreak{}
    \nobreak\extramarks{Problem \arabic{#1} (continued)}{Problem \arabic{#1} continued on next page\ldots}\nobreak{}
}

\newcommand{\exitProblemHeader}[1]{
    \nobreak\extramarks{Problem \arabic{#1} (continued)}{Problem \arabic{#1} continued on next page\ldots}\nobreak{}
    \stepcounter{#1}
    \nobreak\extramarks{Problem \arabic{#1}}{}\nobreak{}
}

\setcounter{secnumdepth}{0}
\newcounter{partCounter}
\newcounter{homeworkProblemCounter}
\setcounter{homeworkProblemCounter}{1}
\nobreak\extramarks{Problem \arabic{homeworkProblemCounter}}{}\nobreak{}

\newenvironment{homeworkProblem}{
    \section{Problem \arabic{homeworkProblemCounter}}
    \setcounter{partCounter}{1}
    \enterProblemHeader{homeworkProblemCounter}
}{
    \exitProblemHeader{homeworkProblemCounter}
}

%
% Homework Details
%   - Title
%   - Due date
%   - Class
%   - Section/Time
%   - Instructor
%   - Author
%

\newcommand{\hmwkTitle}{Matrix Basics}
\newcommand{\hmwkDueDate}{\today}
\newcommand{\hmwkClass}{Engineering Mathematics Basics}
\newcommand{\hmwkClassTime}{}
\newcommand{\Address}{SSE TongJi University}
\newcommand{\hmwkClassInstructor}{Professor Qingjiang Shi}
\newcommand{\hmwkAuthorName}{Zhicheng Liu}

%
% Title Page
%

\title{
    \vspace{2in}
    \textmd{\textbf{\hmwkClass\\ \hmwkTitle}}\\
    \normalsize\vspace{0.1in}\small{By\ \textit{\hmwkClassInstructor\ \hmwkClassTime }}\\
 %   \vspace{0.1in}\small{\Address \\ Due\ on\ \hmwkDueDate\  }
    \vspace{3in}
}

\author{\textbf{\hmwkAuthorName}}
\date{\today}

\renewcommand{\part}[1]{\textbf{\large Part \Alph{partCounter}}\stepcounter{partCounter}\\}

%
% Various Helper Commands
%

% Useful for algorithms
\newcommand{\alg}[1]{\textsc{\bfseries \footnotesize #1}}

% For derivatives
\newcommand{\deriv}[1]{\frac{\mathrm{d}}{\mathrm{d}x} (#1)}

% For partial derivatives
\newcommand{\pderiv}[2]{\frac{\partial}{\partial #1} (#2)}

% Integral dx
\newcommand{\dx}{\mathrm{d}x}

% Alias for the Solution section header
\newcommand{\solution}{\textbf{\large Solution}}

% Probability commands: Expectation, Variance, Covariance, Bias
\newcommand{\E}{\mathrm{E}}
\newcommand{\Var}{\mathrm{Var}}
\newcommand{\Cov}{\mathrm{Cov}}
\newcommand{\Bias}{\mathrm{Bias}}

\begin{document}
	\maketitle
	\newpage
	\begin{enumerate}
		\item %1
		Let A be the $n \times n$ matrix with $a_{j,j+1}=1,j=1,...,n-1$ and all other elements zero. Represent A as a sum of vector outer products.
		
		solution:
		
		     $A=\sum_{j=1}^{j=n-1}{ I_{j}I_{j} }$
		
		\item%2
		Consider the block matrix
		 $\begin{bmatrix}A&B\\C&D\end{bmatrix}$ with $A \in C^{n \times n}$ nonsingular.
		
		1) Determine $X \in C^{m \times m}$ and $Y \in C^{n \times n}$ such that
		
		\[ \begin{bmatrix} A&B\\C&D \end{bmatrix} = \begin{bmatrix} I&O\\{CA^{-1}}&I \end{bmatrix} \begin{bmatrix} A&B\\O&X \end{bmatrix} (block LU factorization)\] 
		
		and that
		
		\[ \begin{bmatrix} A&B\\C&D \end{bmatrix} = \begin{bmatrix} I&BD^{-1}\\O&I \end{bmatrix} \begin{bmatrix} Y&O\\C&D \end{bmatrix} (block LU factorization)\] 
		
		solution:
		
		prove the first equation:if $X=D-CA^{-1}B$,\[ \begin{bmatrix} I&O\\{CA^{-1}}&I \end{bmatrix} \begin{bmatrix} A&B\\O&X \end{bmatrix}
		= \begin{bmatrix} IA & IB \\ CA^{-1}A & CA^{-1}B+IX \end{bmatrix}
		= \begin{bmatrix} A & B \\ C & CA^{-1}B+X \end{bmatrix}
		=\begin{bmatrix} A&B\\C&D \end{bmatrix}
		\] 
		
		prove the second equation:if $Y=A-BD^{-1}C$,\[ \begin{bmatrix} I&BD^{-1}\\O&I \end{bmatrix} \begin{bmatrix} Y&O\\C&D \end{bmatrix}
		= \begin{bmatrix} IY+BD^{-1}C & BD^{-1}D \\ IC &ID \end{bmatrix}
		= \begin{bmatrix} Y+BD^{-1}C & B \\ C & D \end{bmatrix}
		=\begin{bmatrix} A&B\\C&D \end{bmatrix}
		\] 
		
		2) Prove that
		\begin{gather}
			det(A)det(D-CA^{-1}B)=det(D)det(A-BD^{-1}C)\\
			det(I-CB)=det(I-BC)\\
			det(I-xy^*) = 1-y^*x
		\end{gather}
		where $x, y \in C^n$.
		
		solution:$ $ 
		
		prove the first equation:$ $
		
		\[   \begin{bmatrix} I&O\\{CA^{-1}}&I \end{bmatrix} \begin{bmatrix} A&B\\O&X \end{bmatrix} 
		=\begin{bmatrix} A&B\\C&D \end{bmatrix}
		 = \begin{bmatrix} I&BD^{-1}\\O&I \end{bmatrix} \begin{bmatrix} Y&O\\C&D \end{bmatrix}\] 
		 \[ det(\begin{bmatrix} I&O\\{CA^{-1}}&I \end{bmatrix}) det(\begin{bmatrix} A&B\\O&X \end{bmatrix}) 
		 =det(\begin{bmatrix} I&BD^{-1}\\O&I \end{bmatrix}) det(\begin{bmatrix} Y&O\\C&D \end{bmatrix})\]
		 \[ det(I)det(AX)=det(I)det(YD)\]
		 \[ det(AX)=det(YD)\],$X=D-CA^{-1}B$,$Y=A-BD^{-1}C$
		 
		 \[det(A)det(D-CA^{-1}B)=det(D)det(A-BD^{-1}C)\]
		
		prove the second equation:$ $
		
		if A=D=I,
		
		\[det(I)det(I-CI^{-1}B)=det(I)det(I-BI^{-1}C)\]
		
		\[det(I-CB)=det(I)det(I-BC)\]
		
		prove the last equation:$ $
		
		if C=x,B=$y^{*}$,
		
		$det(I-xy^{*})=det(I)det(I- y^{*}x)$ 
		
		$det(I-xy^{*})=det(I-y^{*}x)$
		
		Because A is a $1 \times 1$ matrix,
		
		$det(I-xy^{*})=1-y^{*}x$
		
		\item%3
		Let $A \in C^{n \times n}$, the trace of A is defined as the sum of its diagonal elements, i.e., $trace(A)=\sum_{i=1}^n{a_{ii}}$.
		
		1) Show that the trace is a linear function, i.e., if $A,B \in C^{n \times n}$ and $\alpha,\beta \in C$, then
		\[ trace(\alpha A+\beta B)=\alpha trace(A)+\beta trace(B).\]
		
		solution:$ $ 
		
			$trace(\alpha A+\beta B)=\sum_{i=1}^n{(\alpha a_{ii}+\beta b_{ii})}=\alpha \sum_{i=1}^n{a_{ii}}+\beta \sum_{i=1}^n{b_{ii}}=\alpha trace(A)+\beta trace(B)$
		
		2) Show that $trace(AB)=trace(BA)$, even though in general $AB \neq BA$.
		
		solution:$ $
		
			$trace(AB)=\sum_{i=1}^n\sum_{k=1}^n{ a_{ik}b_{ki} })$
			
			$trace(BA)=\sum_{i=1}^n\sum_{k=1}^n{ b_{ik}a_{ki} }
			=\sum_{k0=1}^n\sum_{i0=1}^n{ b_{k0i0}a_{i0k0} }
			=\sum_{k0=1}^n\sum_{i0=1}^n{ a_{i0k0}b_{k0i0} }
			=trace(AB)$
		
		3) Show that if $S \in R^{n \times n}$ is skew-symmetric, i.e., $S^T=-S$, the $trace(S)=0$. Prove the converse to this statement or provide a counterexample.
		
		solution:$ $
		
		$trace(S)=trace(S^T)=trace(-S)=-trace(S)$
		
		So,trace(S)=0.
		\[\]counterexample of the converse to this statement:
		
		$S=\begin{bmatrix} 0&1\\0&0 \end{bmatrix},trace(\begin{bmatrix} 0&1\\0&0 \end{bmatrix})=0$
		
		$S^T=(\begin{bmatrix} 0&1\\0&0 \end{bmatrix})^T
		=(\begin{bmatrix} 0&0\\1&0 \end{bmatrix})\neq
		(\begin{bmatrix} 0&-1\\0&0 \end{bmatrix})=-S$
		
		\item%4
		Let $\lambda_1, \lambda_2, ..., \lambda_n$ be the eigenvalues of A. Show that
		\begin{gather}
			\setcounter{equation}{0}
			det(A)=\lambda_1\lambda_2...\lambda_n,\\
			trace(A)=\lambda_1+\lambda_2+...+\lambda_n,\\
			trace(A^k)=\lambda_1^k + \lambda_2^k + ... + \lambda_n^k, k=1,2,...
		\end{gather}
	
	    solution:$ $
	    
	    prove the first equation:$A=C^{-1}
	    \begin{bmatrix} \lambda _1 & \cdots\ &0\\
	    	\vdots &\ddots &\vdots \\
	    	0& \cdots\ &\lambda _n \end{bmatrix}
	    C$
	    
	    $det(A)=det(C^{-1})
	    det(\begin{bmatrix} \lambda _1 & \cdots\ &0\\
	    	\vdots &\ddots &\vdots \\
	    	0& \cdots\ &\lambda _n \end{bmatrix})
	    det(C)$
	    
	    $det(A)=det(C^{-1}C)
	    det(\begin{bmatrix} \lambda _1 & \cdots\ &0\\
	    	\vdots &\ddots &\vdots \\
	    	0& \cdots\ &\lambda _n \end{bmatrix})$
    	
    	$det(A)= \lambda _1  \cdots\  \lambda _n $
    	
    	\[\]prove the second equation:$ $
    	
    	$det(\lambda -A)=(\lambda -\lambda_1)\cdots\ (\lambda -\lambda_n)$
    	
    	the coefficient of the term of $\lambda^{n-1}$ is $-(\lambda_1+\cdots\ +\lambda_n)$
    	
    	$det(\lambda -A)=det(\begin{bmatrix} 
    		\lambda -a_{1,1} & \cdots\ &-a_{1,n} \\
    		\vdots &\ddots &\vdots \\
    		-a_{n,1}& \cdots\ &\lambda -a_{n,n} 
    	\end{bmatrix})$
    
        the coefficient of the term of $\lambda^{n-1}$ is only related to $(\lambda-a_{11})(\lambda-a_{22}) \cdots\ (\lambda-a_{nn})$
        
        so,the coefficient of the term of $\lambda^{n-1}$ is $-(a_1+\cdots\ +a_n)$
        
        so,$-(\lambda_1+\cdots\ +\lambda_n)=-(a_1+\cdots\ +a_n)$
    	
    	\[\]prove the last equation:$ $
    	
    	if $A \alpha =\lambda_1 \alpha ,A^k=\lambda ^k\alpha $
    	
    	$So, \lambda_1^k, \lambda_2^k, ..., \lambda_n^k$ be the eigenvalues of $A^k$.
    	
    	
    	$So,trace(A^k)=\lambda_1^k + \lambda_2^k + ... + \lambda_n^k,k=1,2,...$
		
		\item%5
		1) Let $x_1, x_2, ..., x_k$ be eigenvectors of A. Show that $S = span\{x_1, x_2, ..., x_k\}$ is invariant under A.
		
		solution:$ $
		
		if $y\in S$,
		
		$\exists a_1 ,\cdots\ a_k,y=a_1 x_1+\cdots\ a_k x_k$
		
		$so,Ay=a_1\lambda _1 x_1+\cdots\ +a_k\lambda _k x_k$
		
		so,S is invariant under A.
		
		2) Let $A=\begin{bmatrix} 2&1&0\\0&2&0\\0&0&1\end{bmatrix}$. Show that the space $S=span\{e_1,e_2\}$ is invariant under A and is not spanned by eigenvectors of A. Here $e_1 = \begin{bmatrix} 1\\0\\0 \end{bmatrix}$ and $e_2 = \begin{bmatrix} 0\\1\\0 \end{bmatrix}$.
		
		solution:$ $
		
		$\forall y\in S,$ y=$c_1 e_1 +c_2 e_2$
		
		$Ay=A c_1 e_1 +A c_2 e_2=c_1 A e_1 +c_2 A e_2
		=c_1 \begin{bmatrix} 2\\0\\0  \end{bmatrix}
		+c_1 \begin{bmatrix} 1\\2\\0  \end{bmatrix}
		=(2c_1+c_2)e_1+2c_2 e_2$
		
		so,S is invariant under A.
		
		$det(\lambda E-A)=0$ can derive $\lambda _1=\lambda _2 =2,\lambda _3=1$
		
		$(2E-A)\alpha =0$ can derive $\alpha _1=k_1\begin{bmatrix} 1\\0\\0  \end{bmatrix} $
		
		$(E-A)\alpha =0$ can derive $\alpha _2=k_2\begin{bmatrix} 0\\0\\1  \end{bmatrix} $
		
		$e_2=b_1 \begin{bmatrix} 1\\0\\0  \end{bmatrix} +b_2 \begin{bmatrix} 0\\0\\1  \end{bmatrix} $ can derive that $b_1,b_2$ has no solution.
		
		so,S is not spanned by eigenvectors of A.
		
		\item%6
		Verify that $\parallel xy^*\parallel_F = \parallel xy^* \parallel_2 = \parallel x \parallel_2 \parallel y \parallel_2$ for any $x, y \in C^n$.
		
		solution:$ $
		
		$\parallel xy^*\parallel ^2_F =trace((yx^*)(xy^*))
		=(x^*x)trace(yy^*)=\sum_{i=1}^n{x_{i}^2}\sum_{j=1}^n{y_{j}^2}$
		
		$\parallel xy^*\parallel _2
		=\max \limits_{z}((\parallel xy^*z\parallel _2^2)/(\parallel z\parallel _2^2))
		=\max \limits_{z}((\lvert y^*z \rvert \parallel x\parallel _2^2)/(\parallel z\parallel _2^2))
		=\parallel y\parallel _2^2\parallel ky\parallel _2^2\parallel x\parallel _2^2/\parallel ky\parallel _2^2=\parallel x\parallel _2^2 \parallel y\parallel _2^2$,k is a number.
		
		\item%7
		Let $A \in C^{n \times m}$. Show that $\parallel A \parallel_2 \leq \parallel A \parallel_F$. (Recall that $trace(B)=\sum_i\lambda_i(B)$).
		
		solution:$ $
		
		$\parallel A\parallel _F^2=trace(A^*A)$
		
		$\parallel A\parallel _2^2=\max \limits_{x}\frac{\parallel Ax\parallel _2^2}{\parallel x\parallel _2^2}= \max \limits_{i}\lambda _i$
		
		$trace(A^*A)=\sum_{i=1}^n{\lambda _i}> \max \limits_{i}\lambda _i$
		
		\item%8
		For  $A \in C^{n \times m}$, show that
		\begin{gather}
			\setcounter{equation}{0}
			\parallel A \parallel_1 = \underset{x \neq 0}{max}\frac{\parallel Ax \parallel_1}{\parallel x \parallel_1} = \underset{1 \le j \le n}{max}{\sum_{i = 1}^{m} \mid a_{ij} \mid} \\
			\parallel A \parallel_\infty = \underset{x \neq 0}{max}\frac{\parallel Ax \parallel_\infty}{\parallel x \parallel_\infty} = \underset{1 \le j \le n}{max}{\sum_{i = 1}^{m} \mid a_{ij} \mid}
		\end{gather}
	
	    solution:$ $%8
	    
	    $\parallel A \parallel_1
	    = \max \limits_{x\neq 0}\frac{\parallel Ax \parallel_1}{\parallel x \parallel_1}
	    = \max \limits_{x\neq 0}\frac{\sum _i \lvert \sum _j a_{ij}x_i \rvert}
	                                 {\sum _i \lvert x_i \rvert }
	 \leq \max \limits_{x\neq 0}\frac{\sum _i \sum _j \lvert a_{ij}x_i \rvert}
	                                 {\sum _i \lvert x_i \rvert }
	 \leq \max \limits_{x\neq 0}\frac{(\max \limits_{j}\sum _i(a_{ij})) \sum _j \lvert x_i \rvert
	 }{\sum _i \lvert x_i \rvert }
     =\max \limits_{j}\sum _i(a_{ij}) $
	    
	    The proof of the second equation is similar.
		
		\item%9
		Let $A, B \in R^{m \times n}$. Recall that $\parallel A \parallel_F^2 = trace(A^TA)$ and that for any matrix $D$ for which the product $AD$ is defined, $trace(AD)=trace(DA)$.
		
		1) Show that the $Q$ that minimizes $\parallel A - QB \parallel_F$ over all choices of orthogonal $Q$ also maximizes $trace(A^TQB)$.
		
		solution:$ $%9
		
		$\parallel A-QB \parallel _F^2
		=trace((A-BQ)^T(A-QB))
		=trace((A^T-B^T Q^T)(A-QB))
		=trace(A^T A+B^T Q^T Q B -A^T QB-B^T Q^T A)
		=trace(A^T A+B^T B)-2trace(A^T QB)$
		
		$trace(A^T A+B^T B)$ is constant.
		
		so,the $Q$ that minimizes $\parallel A - QB \parallel_F$ over all choices of orthogonal $Q$ also maximizes $trace(A^TQB)$
		
		2) Suppose that the $SVD$ of the $m \times m$ matrix $BA^T$ is $U\Sigma V^T$, where $U$ and $V$ are $m \times m$ and orthogonal and $\Sigma$ is diagonal with diagonal entries $\sigma_1 \ge ... \ge \sigma_m \ge 0$. Define $Z = V^TQU$. Use these definitions and $(i)$ to show that
		\[trace(A^TQB) = trace(Z\Sigma) \le \sum_{i=1}^{m}{\sigma_i}\].
		
		solution:$ $
		
		$trace(A^T QB)=trace(QB A^T)=trace(QU \Sigma V^T)=trace(V^T QU\Sigma )=trace(Z\Sigma )$
		
		Z is orthogonal and diagonal,hence $z_{ij}$=0 or 1 or -1,
		\label{{\tiny key}}
		$trace(Z\Sigma )= \sum \limits_i z_{ii} \sigma _i \leq \sum \limits_i \sigma _i$
		
		3) Identify the choice of $Q$ that gives equality in the bound of $(ii)$.
		
		solution:$ $
		
		if $Q=VU^T$,
		
		$Z=V^TQU=V^TVU^TU=V^{-1}VU^{-1}U=II=I$
		
		$trace(Z\Sigma )=trace(\Sigma )=\sum \limits_i \sigma _i$
		
		4) Carefully state a theorem summarizing the solution to the minimization of $\parallel A - QB \parallel_F$.
		
		solution:$ $
		
		$Q=VU^T,Q^T=Q^{-1},\parallel A - QB \parallel_F$ is minimized by Q $\Rightarrow$ 
		$BA^T=U \Sigma V^T$is an SVD.
		
		\item
		Let $A \in C^{m \times n}$.
		
		1) Show that $(range(A))^\perp = null(A^*)$ and that $(null(A))^\perp = range(A^*)$.
		
		solution:$ $
		
		$A=U  \Sigma V^*
		,r=Rank(A)
		\Rightarrow null(A^*)
		$=$span\{u_{r+1},\cdots\ ,u_m\},range(A)=span\{u_1,\cdots\ ,u_r\}
		\Rightarrow ( range(A) )^\perp =null(A)
		\Rightarrow range(A)=(null(A))^\perp $
		
		2) Show respectively that $AA^+, A^+A, I-A^+A and I-AA^+$ are the orthogonal projectors onto $range(A)$, $range(A^*), null(A) and null(A^*)$.
		
		solution:$ $
		
		First part:$ $
		
		$(AA^+)^2=(AA^+A)A^+=AA^+$
		
		$\forall x\in range(A),x=Ay=AA^+Ay=AA^+x 
		\Rightarrow x\in range(AA^+)$
		
		$\forall x\in range(AA^+),x=AA^+y=A(A^+y)
		\Rightarrow x\in range(A)$
		
		hence $AA^+$ are the orthogonal projectors onto $range(A)$.
		
		Second part:$ $
		
		By first part,$A^*(A^*)^+$ are the orthogonal projectors onto $range(A^*)$.
		
		hence $A^+A$ are the orthogonal projectors onto $range(A^*)$.
		
		Third part:$ $
		
		By first part,$I-AA^+$ are the orthogonal projectors onto $range(A^\perp )=null(A)$.
		
		hence $I-AA^+$ are the orthogonal projectors onto $null(A)$.
		
		Last part:$ $
		
		By second part,$I-A^+A$ are the orthogonal projectors onto $range((A^*)^\perp )=null(A^*)$.
		
		hence $I-AA^+$ are the orthogonal projectors onto $null(A^*)$.
		
		3) Suppose $rank(A)=r$, let $A = U\Sigma V^*$ with $ U = \lbrack U_1 \quad U_2 \rbrack$ and	$V = \lbrack V_1 \quad V_2\rbrack$ unitary and partitioned, such that $U_1 \in C^{m \times r}$ and $V_1 \in C^{n \times r}$. Show that $U_1U_1^*, V_1V_1^*, V_2V_2^*$ and $U_2U_2^*$ are the orthogonal projectors onto $range(A), range(A^*), null(A)$ and $null(A^*)$, respectively.
		
		solution:$ $
		
		$AA^+=U\Sigma V^*(V\Sigma ^+ U^*)
		=U
		 \begin{bmatrix} I_r&0\\0&0 \end{bmatrix} 
		 U^*
		 =U1U1^*$
		 
		 hence $U1U1^*$ are the orthogonal projectors onto $range(A)$.
		 
		 other parts are similar. 
		
	\end{enumerate}
\end{document}